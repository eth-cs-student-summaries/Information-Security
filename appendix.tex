%
% Appendix
%
\appendix

\part{Appendix}

\section{Notation}

Below is a non-exhaustive list of notations and symbols which are used throughout this document and which the reader may not have seen before:

\begin{itemize}
    \item $||$ --- concatenation operator
    \item $\oplus$ --- xor operator
    \item $x \leftarrow G$ --- $x$ is chosen from group $G$
    \item PPT --- probabilistic polynomial time
\end{itemize}


\subsection{Message construct notation}

\begin{table}[h]
\centering
\begin{tabular}{ll}
%\hline
Message & $\{ M \}$ \\
Name & $A$, $B$ or $Alice$, $Bob$ \\
Asymmetric keys &  $A$ has public key $K_A$ and private key $K_A^{-1}$ \\
Symmetric keys & $K_{AB}$ shared by $A$ and $B$ \\
Encryption &  asymmetric $\{ M \}_{K_A}$ or symmetric $\{ M \}_{K_{AB}}$ \\
Signing &  $\{ M \}_{K_A^{-1}}$\\
Nonces &  $N_A$ ("number used once", freshly created by $A$)\\
Timestamps & $T$ (eg expiration) \\
%\hline
\end{tabular}
\end{table}

\section{Reading Tamarin Graphs}

Actions of the same agent are coloured with the same shade of green.

The boxes mirror a labeled multiset rewriting rule $l \xrightarrow{a} r$. Thus the three rows, top to bottom, mean:

\begin{enumerate}
    \item state fact before rewriting
    \item action fact/event
    \item state fact after rewriting
\end{enumerate}

Arrows have the following colour-coding:

\begin{itemize}
    \item grey   $\longrightarrow$ using persistent facts
    \item black  $\longrightarrow$ using linear facts
    \item orange $\longrightarrow$ sending a message into to the network, i.e. to the adversary
    \item dotted $\longrightarrow$ attacker using a rule
\end{itemize}

\pagebreak

Other syntactical things to note:

\begin{itemize}
    \item \$ -- prefixes public values (e.g. agent names, state facts. \\
                Notabene: recipients of messages containing values prefixed with \$ do check whether the value is indeed public.
    \item ! -- prefixes persistent facts
    \item $\sim$ -- prefixes fresh facts
\end{itemize}

\underline{Exam relevance:}\\
It is sufficient to be able to read a Tamarin graph. You neither need be able to write Tamarin theory from scratch, nor to know any detailed syntax -- but you are expected to be familiar with the Dolev-Yao attacker model and be able to reason about security protocols and their execution. See past exams for examples.


%\bibliographystyle{plain}
%\bibliography{references}